\chapter{The Eclipse Plug-in}
\label{EclipsePlugin}

Since OpenJML operates on Java files, it is natural that it be integrated into the Eclipse IDE for Java.
OpenJML provides a conventional Eclipse plug-in that encapsulates the OpenJML command-line tool and integrates it
with the Eclipse Java development environment.

\section{Installation and System Requirements}
Your system must have the following:
\begin{itemize}
\item A Java 1.7 JRE as described in section \ref{Running}. This must be the JRE in use in the environment in which Eclipse is invoked. If you start Eclipse by a command in a shell, it is straightforward to make sure that the correct Java JRE is defined in that shell.  However, if you start Eclipse by, for example, double-clicking a desktop icon, then you must ensure that the Java 1.7 JRE is set by the system at startup.
\item Eclipse 4.2 or later
\end{itemize}

Installation of the plug-in follows the conventional Eclipse procedure.
\begin{itemize}
\item Invoke the "Install New Software" dialog under the Eclipse "Help" menubar item.
\item "Add" a new location, giving the URL \url{http://jmlspecs.sourceforge.net/openjml-updatesite} and some name of your choice (e.g. OpenJML).
\item Select the "OpenJML" category and push "Next"
\item Proceed through the rest of the wizard dialogs to install OpenJML.
\item Restart Eclipse when asked to obtain full functionality.
\end{itemize}

If the plug-in is successfully installed, a yellow coffee cup (the JML icon) will appear in the menubar (along with other menubar items).
The installation will fail (without obvious error messages), if the underlying Java VM is not a suitable Java 1.7 VM.

\section{GUI Features}

\textit{This section will be added later.} %% TBD
