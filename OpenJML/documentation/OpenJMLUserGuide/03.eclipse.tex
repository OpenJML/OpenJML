\chapter{The Eclipse Plug-in}
\label{EclipsePlugin}

Since OpenJML operates on Java files, it is natural that it be integrated into the Eclipse IDE for Java.
OpenJML provides a conventional Eclipse plug-in that encapsulates the OpenJML command-line tool and integrates it
with the Eclipse Java development environment. The plug-in also provides GUI functionality for working with JML specifications.

\section{Installation and System Requirements}
Your system must have the following:
\begin{itemize}
\item A Java 1.7 JRE as described in section \ref{Running}. This must be the JRE in use in the environment in which Eclipse is invoked. If you start Eclipse by a command in a shell, it is straightforward to make sure that the correct Java JRE is defined in that shell.  However, if you start Eclipse by, for example, double-clicking a desktop icon, then you must ensure that the Java 1.7 JRE is set by the system at startup.
\item Eclipse 4.2 or later. The OpenJML plug-in is currently tested for the Juno, Kepler, and Luna releases of Eclipse.
\item One or more SMT solvers, if the static checking functionality will be used.
\end{itemize}

Installation of the plug-in follows the conventional Eclipse procedure.
\begin{itemize}
\item Invoke the "Install New Software" dialog under the Eclipse "Help" menubar item.
\item "Add" a new location, giving the URL \url{http://jmlspecs.sourceforge.net/openjml-updatesite} and some name of your choice (e.g. OpenJML).
\item Select the "OpenJML" category and push "Next"
\item Proceed through the rest of the wizard dialogs to install OpenJML.
\item Restart Eclipse when asked to obtain full functionality.
\end{itemize}

If the plug-in is successfully installed, a yellow coffee cup (the JML icon) will appear in the toolbar (along with other menubar/toolbar items).
The installation will fail (without obvious error messages), if the underlying Java VM is not a suitable Java 1.7 VM.

\section{GUI Features}

Note that the JML logo is a JML-decorated yellow coffee cup; this logo is associated with various GUI elements.
 
\subsection{Commands}
The OpenJML plug-in adds a number of commands. These are visible in the Preferences>>General>>Keys dialog. All the OpenJML commands explicitly added by the OpenJML plug-in are in the 'JML' category. Some commands are automatically added by Eclipse and are in other categories; for example, Eclipse automatically adds commands to open each individual Preference Page and each kind of View You can sort the table of commands by category and you can filter the table, in order to show just those commands related to JML. Also, this dialog allows binding a keyboard key-combination to a command, as is the case for all Eclipse commands.

The commands are listed in Table~\ref{Tab:commands}, with forward references to more detailed discussion.
\begin{itemize}
\item Add to JML specs path (\S\ref{TBD}). Allows editing the specspath 
\item Clear All Results (\S\ref{TBD}). Deletes all results of static checking operations 
\item Clear Selected Results (\S\ref{TBD}). Deletes some of the results of static checking 
\item Delete JML Markers (\S\ref{TBD}). Deletes all JML markers and highlighting on selected resources 
\item Disable JML on the project (\S\ref{TBD}). 
\item Edit JML Source/Specs Paths (\S\ref{TBD}). 
\item Enable JML on the project (\S\ref{TBD}). 
\item Generate JML doc (\S\ref{TBD}). 
\item insert \bs result (\S\ref{TBD}). 
\item insert .... (\S\ref{TBD}). 
\item Open a Specifications Editor (\S\ref{TBD}). 
\item Open the ESC Results View (\S\ref{TBD}). 
\item RAC - ... (\S\ref{TBD}). 
\item Remove from JML specspath (\S\ref{TBD}). 
\item Rerun Static Check (\S\ref{TBD}). 
\item Show Counterexample (\S\ref{TBD}). 
\item Show Counterexample Value (\S\ref{TBD}). 
\item Show Detailed Proof Attempt Information (\S\ref{TBD}). 
\item Show ESC Result Information (\S\ref{TBD}). 
\item Show JML paths (\S\ref{TBD}). 
\item Show JML Specifications (\S\ref{TBD}). 
\item Static Check (ESC) (\S\ref{TBD}). 
\item Typecheck JML (\S\ref{TBD}). Performs syntax, parsing and typechecking on selected projects, folders, and files.
\item Show In ... (\S\ref{TBD}). 
\item Show View (OpenJML Static Checks) (\S\ref{TBD}). 
\item Show View (OpenJML Trace) (\S\ref{TBD}). 
\item Preferences (OpenJML > OpenJML Solvers) (\S\ref{TBD}). Opens the OpenJML Solvers subpage.
Preferences (OpenJML) (\S\ref{TBD}). Opens the OpenJML Preferences page.
\end{itemize}



\subsection{Menubar additions}
The main Eclipse menubar contains an additional menu titled 'JML'. It contains the following submenu items; typically the action for a menu item is the similarly named command.
Menu items are also added in the following Context menus (context menus are available by right-clicking):
\begin{itemize}[noitemsep,nolistsep]
\item Context menu on elements of the Eclipse Package Explorer View, Project Explorer View, and Navigator View
\item Context menu on elements of the Outline View
\item Context menu within an editor
\item ... Problem View...
\item ... OpenJML Proof view ...
\end{itemize}
Individual menu items may be disabled when they are not applicable. For instance, some items are enabled only when something is selected, some only when exactly one apporpriate item is selected, some only when a method is selected, etc.
%\begin{itemize}
%\item Typecheck JML - executes the 'Typecheck JML' command
%\item TBD - MORE ...
%\end{itemize}

\subsection{Toolbar additions}
The OpenJML plug-in adds three toolbar items; clicking the toolbar item executes a corresponding command.
\begin{itemize}
\item the JML coffee cup logo - executes the 'Typecheck JML' command
\item ESC - executes the 'Static Check (ESC)' command
\item RAC - executes the 'RAC - compile selected' command
\end{itemize}

\subsection{OpenJML Problems, Markers and highlights}

TBD

\subsection{OpenJML console and error log}
The plug-in adds an additional type of console. Eclipse has a Console View that manages the consoles for various plug-ins. The OpenJML Console is created automatically when output is generated. General informational output is sent to the console; errors and warnings are placed in the console and shown in pop-up dialog boxes. Egregious errors are also logged in the Eclipse Error Log.

\textit{Note: Currently the OpenJML Console is not listed in the list of types of Consoles available in Console View menu.}

\subsection{Preferences}

The plug-in adds dialogs for setting OpenJML options. These are workspace preferences, affecting all projects in the workspace. There are two Preference pages:
\begin{itemize}[noitemsep,nolistsep]
\item A top-level page named 'OpenJML' found in the top-level list of Eclipse preference pages. This page allows setting options that would otherwise be set on the command-line. These is also one UI option, named 'UI verbose', that enables verbose output to the OpenJML console about actions within the UI code. 
\item A sub-page named 'Solvers'. (Click the turnstile next to 'OpenJML' in the list of Preference pages to see the subpage). These preferences enable registering SMT solvers and setting the default solver to use.
\end{itemize}

There are currently no project-level preferences within the OpenJML plug-in.

\subsection{Editor embellishments}

TBD - fill out

\begin{itemize}
\item counterexample hovers
\item quick fix proposals
\item contet-sensitive completions
\item insertions
\end{itemize}

\subsection{OpenJML Views}

TBD

\subsection{Help}
There is a 'JML' entry in the table of contents under the Eclipse Help menu item. It provides an online user guide to JML and OpenJML.

\textit{Note - the current information available under Help is outdated and will be replaced by this manual.}

\subsection{Other GUI elements}
\begin{itemize}
\item OpenJML decoration - A decoration is applied to names of projects in the Package Explorer View for which OpenJML has been enabled (cf. \S\ref{TBD}). The decoration is a miniature JML logo on the upper-right of the folder icon, covering the 'J' symbol that indicates an Eclipse Java project.
\item The plug-in defines a JML Nature and a JML Builder. The Nature is associated with a project precisely when JML is enabled for the project. The Builder performs automatic type-checking.  (cf. \S\ref{TBD})
\item \texttt{.jml} suffix. The plug-in adds a content type that associates the \texttt{.jml} filename suffix with the Java editor. This makes the Java editor the default editor for .jml files. 

\textit{Note - so we still get spurious errors on jml files?} % TBD

\item Internationalization. TBD...
\item classpath intializer. TBD ...
\item definition as an Eclipse project. TBD ...
\item Open JML Perspective. TBD ...
\end{itemize}